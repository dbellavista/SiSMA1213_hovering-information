\section{Introduction}

\subsection{Vision}

The hovering information is an information dissemination service working in an
dynamic infrastructure-free environment with a self-organizing behaviour; a MAS\cite{OEOP,epmas}
approach may offer a sound paradigm for both hovering information
implementation and simulation. The simulation implies the design of a
\emph{social} system, where people - hovering information users - move in an
environment with different and non-random behaviour. From the simulation
results, an analysis of the resulting dynamic network can lead to additional
consideration and information that may help understanding and defining service
properties and requirements.

In section~\ref{sec:design} the system is designed using the \emph{SODA}\cite{soda}
methodology. Section~\ref{sec:implementation} describes the implementation
techniques and strategies,

\subsection{Hovering Information System}

\emph{Hovering Information} is a geo-localized information dissemination
service, proposed in \cite{hover}, able to work without a centralized
infrastructure. The service is aimed to mobile users capable of peer-to-peer
communication and geo-localization. The hovering information system is composed
by two main components: mobile nodes and pieces of hovering information.

\emph{Mobile nodes} are components moving into the environment with a limited
communication range, capable of communicate to peers, discover neighbors,
access and store (inside a limited buffer) pieces of hovering information. A
mobile node is assumed able to determinate its geographic position, speed and
direction.

\emph{Pieces of hovering information} are data that have to \emph{survive}
inside a circular area centered at a location called \emph{anchor location} and
having a radius called \emph{anchor radius}. The survivability goal of a piece
of hovering information is achieved moving or replicating the piece itself
through the mobile nodes. A piece of hovering information may have some
policies controlling the movement between nodes.

In an hovering information system, three main requirements may be defined for each
piece of hovering information \cite{hover}:
\begin{description}
	\item[Survivability:] a piece of hovering information is alive at some time
		$t$, if there is at leas one node hosting a replica of this information.
		\\
		$survivability = \frac{alive\_time}{total\_time}$
	\item[Availability:] a piece of hovering information is available at some time
		$t$, if there is at least one node in its anchor area hosting a replica of
		this information.
		\\
		$avaiability = \frac{avaiable\_time}{total\_time}$
	\item[Accessibility:] a piece of hovering information is accessible by a node
		at some time $t$, if the node is able to get this information; therefore, a
		replica exists in the node communication range.
		\\
		$accessibility = \frac{replica\_covered\_area}{anchor\_area}$
\end{description}
