\section{MAS Implementation}
\label{sec:implementation}

The goal of this project is to simulate and analyze the
hovering system used by the social system. In order to
reduce the technological gap between the SODA process and
the implementation, I chosen to use \emph{Jason+CArtAgO}\cite{cartago,jaslide,cartex}.
infrastructure.

\subsection{Initialization}

Prior to simulation, the sistem must be initialized. I couldn't find the
required flexibility in the \code{mas2j} files, so I decided to create a new
Agent, called \emph{initiator}, which reads a configuration file and create
agents, workspaces and common artifacts specifying the needed configuration.
Artifacts are configured by setting the proper initialization parameters, while
agents are configured by telling them what they have to believe (using the
\emph{AgentSpeak} \code{tell} primitive).

The \emph{initiator} creates the person agents and inserts them into the
environment. For each person, a new mobile agent is created. Each person
searches the \emph{MobileUI Artifact}, which represent the mobile device
itself. This artifact is created by the mobile node agent, thus, when the
person agent finds it, both the mobile node and the person are ready to begin.

When the system is about to start, the \emph{initiator} sends an
\code{achieve(start)} to all the person agent. Each person agent turn on their
mobile nodes using a new operation of \emph{MobileUI Artifact}:
\code{StartDevice}, which is linked to the \emph{MobileUIInterface Artifact}
and manifest itself to the mobile agent.

The last initialization operation is the \emph{dissemination}, that is, to
create the initial hovering agent inside some mobile nodes. For now the only
dissemination method is the random dissemination, which assign to each mobile
node an hovering agent with a certain probability.

\subsection{Environment Artifact}

The environment artifact represent the world. Every Body Artifact,
MobileResource and the SimulationArtifact is linked to it. The Environment
Artifact provides operation for insert, sense and move people into the
environment. The \emph{MobileResource} is linked to the Environment Artifact
for discovering neighbors, obtaining the position and sending and receiving
messages.

\subsection{Messaging System}

For the communication, I have chosen to use the \emph{AgentSpeak} primitives
for communication between hovering agents and their host and to use a custom
message-passing architecture for communicating with neighbors. Doing this, I
assumed a simple dispatch-based architecture, easily implementable in real
mobile devices. The messaging system is realized as two new operation:
\code{SendMessage} and \code{ReceiveMessage} into the \emph{MobileResource
Artifat}.

Thanks to the \code{await} method of \emph{CArtAgO}, the receive message
operation blocks the execution flow, completing only when a new message is
available.

\subsection{Obtaining the position}

Obtaining the position is an important task and must be performed continuously.
I have added a new operation \code{StartObtainingPosition} in
\emph{MobileResource Artifact}: using the \code{await} method, the artifact
polls the position, manifesting itself to the agent with the new position.


\subsection{Social System}

People have to move around the world, basing on their environment perception.
In the real world, a person has eyes, through which can recognize other people
and point of interest and approximate their distance.  This can be seen as a
light sensor sensing the physical environment, which has a source producing
light that is reflected by people and points of interest.

With the \code{Sense} operation, each person obtain a list of other people and
points of interest. Behaviors can be build from here.

For now, the only implemented behavior is a sort of random walk: a person chose
a random position and reach it at a random speed.

\subsection{Hovering Survive Strategy}

The behavior of hovering agents depends on various factor. I have decided to
rise the abstraction level by creating the concept of \emph{Defcon}. At defcon
$5$ and $4$, an hovering agent thinks it's safe and just monitor it's position,
speed and so on. At defcon $3$, the hovering starts probing the neighbors,
creating a list of potential destination for landing or cloning operation. At
defcon $2$, it tries to land on a mobile node closer to the anchor area. On
defcon $1$ it clone itself on the best neighbor.

For now the defcon level is computed by a linear combination of two reversed
exponential function. With values close to $1$, the defcon tend to be higher,
while with values closer to $0$, the defcon is lower. The function may depend
on various factors. I have implemented:

\[
  v = e^{-\frac{Distance}{DeathZoneLimit}*ZoneFactor} * e^{\frac{NumNeighbors}{10}}
\]

\[
  ZoneFactor = \begin{cases}
    1, & \mbox{if } \mbox{ok zone} \\
    2, & \mbox{if } \mbox{warning zone} \\
    3, & \mbox{if } \mbox{violent zone} \\
    4, & \mbox{if } \mbox{death zone} \\
    \end{cases}
\]

\[
  DefconLevel = \begin{cases}
    1, & \mbox{if } 0.8  \leq v \\
    2, & \mbox{if } 0.68 \leq v < 0.8  \\
    3, & \mbox{if } 0.42 \leq v < 0.68  \\
    4, & \mbox{if } 0.35 \leq v < 0.42  \\
    5, & \mbox{if } 0.0  \leq v < 0.35  \\
    \end{cases}
\]

Using the defcon level, the hovering agent can limit its interaction basing on
various factors.

\subsection{Hovering Landing and Cloning operations}

In the real world, a landing operation may consist in the agent serializing,
which transfer itself to the new mobile node, which will resume it. On the
other hand, the clone operation should send the agent source and an initial
sets of believes to the target mobile node. The new host will create a new
agent and \emph{tell} him the initial believes.

I have implemented a similar mechanism, but of course the hover agent doesn't
send code or the serialized itself, because they are in the same local system.

When an hovering information wants to land on another mobile node, after asking
the permission, it sends a message, tells the old host that it's dead (using
the Arakiri Speech) and waits for the resume of the new host. It may happen
that the communication fails during the landing. In this case the hovering
agent accepts it's fate and dies alone.

\subsection{Odds and Ends}

The JASON complexity raised, and I have faced some problems, without knowing
the exact cause. They are surely due to a some implementation errors, but I
wasn't able to spot them and the debugging wasn't so helpful. 

This and that lead me to alterate and restore some SODA interaction. For
instance the hovering information tells (and untells) its data to the mobile
node (it's the reverse situation of the \emph{GetData Speech}).

Moreover I haven't implemented some features, such as smarter movement for
people, different and better dissemination process, landing and cloning basing
on availability and accessibility of the hovering information and system
analysis. All those features (except mayhaps for people movement) are easily
implementable using the data available to each agent, but I have give up due to
some random and frustrating errors.
