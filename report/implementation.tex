\section{MAS Implementation}

The goal of this project is to simulate and analyze the
hovering system used by the social system. In order to
reduce the technological gap between the SODA process and
the implementation, I chosen to use \emph{Jason+CArtAgO}
infrastructure.

\subsection{Initialization}
Prior to simulation, the sistem must be initialized. I couldn't find the
required flexibility in the \emph{mas2j} files, so I decided to create a new
Agent, called \emph{initiator}, which reads a configuration from a file and
create agents, workspaces and common artifacts specifying the needed configuration.
Artifacts are configured by setting the proper initialization parameters, while
agents are configured by telling them what they have to believe (using the
Jason \emph{tell} internal operation).

\subsection{Social System}
People have to move around the world, basing on
their environment perception. In the real world,
a person has eyes, through which can recognize other
people and point of interest and approximate their distance.
This can be seen as a light sensor sensing the
physical environment, which has a source producing light
that is reflected by people and points of interest.

From a simplified \emph{artifacful} points of view, the
environment itself is an artifact linked to people's
sensors and actuator. The environment knows the position of
every people, updating its internal state when a person moves
and returning it when the sense operation is executed.

During the initialization phase, a person agent gets to know
its initial position in the environment and, just before to start,
it can enter the environment in the right initial position.
