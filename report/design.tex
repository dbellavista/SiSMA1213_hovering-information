\section{Hovering Information and Social System analysis and design}
\label{sec:design}

The system should implement the hovering information system working inside
a social environment. Mobile nodes are owned by people, who move inside an
environment composed by anchors, that is locations where pieces of hovering
information are present. Anchors are usually bound to points of interest, but
in a more general way hovering information can be dynamically created by people.

Mobile nodes lose power and may have not enough energy to supply the whole
function.  In that case some mobile node features may be limited such as
information storage, communication, etc..

The system should simulate an hovering information usage, inside an
environment.  People, carrying mobile nodes, have to walk with different
behaviour, emulating movements inside an area composed by points of interest.
The simulation should gather periodic data about hovering information status
and properties (i.e.\ availability, survivability and accessibility), nodes
position and communications link.

\subsection{Preliminary analysis}

Requirements implicate that the system is composed by three sub-systems:
\begin{enumerate}
	\item Simulator (graphic interface and data analyzer).
	\item Hovering information (composed by pieces of hovering information and
		mobile nodes).
	\item Social system (a group of hovering information users moving into
		the environment).
\end{enumerate}

Aside from required interfaces, these subsystems can be designed independently
from each other and each of them as different users. A simulator user may want
to create people and assign a behaviour, nodes and initial hovering information.
The simulator itself is the user of the social system and a single person inside
the latter system is an user of the hovering information service.

Social system requires people to move with different behaviours; a possible
solution is to assign \emph{social roles} with different behavior pattern.
Taking the cue from \cite{human} some roles can be defined:

\begin{description}
	\item[Guard:] a person who walks following a predefined path.
	\item[Employee:] a person who resides -works- near a certain point of interest.
	\item[Ant:] a person who performs a random walk but he's influenced in a
		certain manner by other people.
	\item[Group:] some people walking together, using one of the previous
		behaviour.
\end{description}

\subsection{Requirements Analysis}

The subsystem \emph{Simulator} doesn't require independent control or
intelligence, so the simulator interface can be assumed as a \emph{Legacy
System}, already present in the environment.

\subsubsection*{Requirements Tables}

\begin{table}[H]
	\centering
	\begin{tabular}{|p{4cm}|p{8cm}|}
			\hline
			\textbf{Actor} & \textbf{Description} \\
			\hline
			Simulation Analyst & The simulation data analyzer. \\
			\hline
			Person & User of a mobile node. \\
			\hline
			Mobile node & Hovering information low-level user: discover, access and
			create the infrastructure for the pieces of hovering information. \\
			\hline
			Piece of Hovering Information & Hovering information instance. \\
			\hline
		\end{tabular}
	\caption{Actor table $(C)Ac_t$}
	\label{tab:cact}
\end{table}

\begin{table}[H]
	\centering
	\begin{tabular}{|p{4cm}|p{8cm}|}
			\hline
			\textbf{Requirement} & \textbf{Description} \\
			\hline
			Access Information & Access all the information that reside inside an
			anchor area. \\
			\hline
			Create Information & Creates a new hovering information. \\
			\hline
			Obtain System Data & Known the current data (position, information access,
			etc.) of all the system component. \\
			\hline
			Manage Simulation & Manage the simulation. \\
			\hline
			Walk & Move into the environment. \\
			\hline
			Survive & Stay alive. \\
			\hline
			Be available & Be available inside the anchor area. \\
			\hline
			Maintain accessibility & Cover the maximum possible area inside the
			anchor area. \\
			\hline
			Define initial parameters & Simulator users should be able to specify
			initial simulation parameters (environment, people, information) \\
			\hline
		\end{tabular}
	\caption{Requirement table $(C)Re_t$}
	\label{tab:cact}
\end{table}

\begin{table}[H]
	\centering
	\begin{tabular}{|p{4cm}|p{8cm}|}
			\hline
			\textbf{Actor} & \textbf{Requirement} \\
			\hline
			Person & Access Information, Create Information, Walk. \\
			\hline
			Simulation Analyst & Obtain System Data, Manage Simulation, Define
			initial parameters. \\
			\hline
			Mobile node & Access Information, Create Information. \\
			\hline
			Piece of Hovering Information & Survive, Be available, Maintain
			accessibility. \\
			\hline
		\end{tabular}
	\caption{Actor-Requirement table $(C)AR_t$}
	\label{tab:cart}
\end{table}

\subsubsection*{Domain Tables}

\begin{table}[H]
	\centering
	\begin{tabular}{|p{4cm}|p{8cm}|}
			\hline
			\textbf{External Environment} & \textbf{Legacy system} \\
			\hline
			External & Simulator UI. \\
			\hline
		\end{tabular}
	\caption{External Environment-Legacy System table $(C)EELS_t$}
	\label{tab:ceelst}
\end{table}

\begin{table}[H]
	\centering
	\begin{tabular}{|p{4cm}|p{8cm}|}
			\hline
			\textbf{Legacy System} & \textbf{Description} \\
			\hline
			Simulator UI & Simulation output interface: show simulation data. \\
			\hline
		\end{tabular}
	\caption{Legacy System table $(C)LS_t$}
	\label{tab:clst}
\end{table}

\subsubsection*{Relations Tables}

\begin{table}[H]
	\centering
	\begin{tabular}{|p{4cm}|p{8cm}|}
			\hline
			\textbf{Relation} & \textbf{Description} \\
			\hline
			Simulator Data & make relevant information available to the Simulator UI.
			\\
			\hline
			Define Before Simulate & initial environment definition should occurs
			before simulation. \\
			\hline
			Create Before Exist & An information has to been created before
			performing any operation. \\
			\hline
			Parameters Input & Simulation parameters have to be inserted into the
			simulation system. \\
			\hline
		\end{tabular}
	\caption{Relation table $(C)Rel_t$}
	\label{tab:crelt}
\end{table}

\begin{table}[H]
	\centering
	\begin{tabular}{|p{4cm}|p{8cm}|}
			\hline
			\textbf{Requirement} & \textbf{Relation} \\
			\hline
			Access Information & Simulator Data, Define Before Simulate, Create
			Before Exist. \\
			\hline
			Create Information & Simulator Data, Define Before Simulate, Create
			Before Exist. \\
			\hline
			Obtain System Data & Simulator Data, Define Before Simulate. \\
			\hline
			Manage Simulation & Simulator Data, Define Before Simulate. \\
			\hline
			Walk & Define Before Simulate. \\
			\hline
			Survive & Define Before Simulate, Create Before Exist. \\
			\hline
			Be available & Define Before Simulate, Create Before Exist. \\
			\hline
			Maintain accessibility & Define Before Simulate, Create Before Exist. \\
			\hline
			Define initial environment & Define Before Simulate, Create Before
			Exist, Parameters Input. \\
			\hline
		\end{tabular}
	\caption{Requirement-Relation table $(C)RR_t$}
	\label{tab:crrt}
\end{table}

\begin{table}[H]
	\centering
	\begin{tabular}{|p{4cm}|p{8cm}|}
			\hline
			\textbf{Legacy-System} & \textbf{Relation} \\
			\hline
			Simulator UI & Simulator Data. \\
			\hline
		\end{tabular}
	\caption{Relation-LegacySystem table $(C)RLS_t$}
	\label{tab:crlst}
\end{table}

\subsection{Analysis}

\subsubsection*{References Tables}

\begin{table}[H]
	\centering
	\begin{tabular}{|p{4cm}|p{8cm}|}
			\hline
			\textbf{Requirement} & \textbf{Task} \\
			\hline
			Access Information & list\_information \newline access\_information \\
			\hline
			Create Information & create\_information \\
			\hline
			Obtain System Data & obtain\_nodes\_information \newline
			obtain\_hovering\_information \newline
			obtain\_communication\_links \\
			\hline
			Manage Simulation & start\newline stop\newline pause \\
			\hline
			Walk & walk\newline. \\
			\hline
			Survive & survive. \\
			\hline
			Be available & be\_available. \\
			\hline
			Maintain accessibility & maintain\_accessibility. \\
			\hline
			Define initial parameters & create\_information\newline create\_people. \\
			\hline
		\end{tabular}
	\caption{Reference Requirement-Task table $(C)RRT_t$}
	\label{tab:crrtt}
\end{table}

\begin{table}[H]
	\centering
	\begin{tabular}{|p{4cm}|p{8cm}|}
			\hline
			\textbf{Requirement} & \textbf{Function} \\
			\hline
			Access Information & communicate\_data\newline show\_information\newline discover\_neighbor \\
			\hline
			Create Information & communicate\_data\newline insert\_information\newline discover\_neighbor \\
			\hline
			Obtain System Data & inquire\_node\newline inquire\_hovering\_information \\
			\hline
			Manage Simulation & render \\
			\hline
			Walk & detect\_people. \\
			\hline
			Survive & replicate\newline jump\newline discover\_neighbor. \\
			\hline
			Be available & replicate\newline jump\newline discover\_neighbor. \\
			\hline
			Maintain accessibility & replicate\newline jump\newline discover\_neighbor. \\
			\hline
			Define initial parameters & accept\_input. \\
			\hline
		\end{tabular}
	\caption{Reference Requirement-Function table $(C)RRF_t$}
	\label{tab:crrft}
\end{table}

\begin{table}[H]
	\centering
	\begin{tabular}{|p{4cm}|p{8cm}|}
			\hline
			\textbf{Requirement} & \textbf{Topology} \\
			\hline
			Access Information & Anchor Area, Communication Range \\
			\hline
			Create Information & Anchor Area, Communication Range \\
			\hline
			Survive & Anchor Area, Communication Range. \\
			\hline
			Be available & Anchor Area, Communication Range. \\
			\hline
			Maintain accessibility & Anchor Area, Communication Range. \\
			\hline
		\end{tabular}
	\caption{Reference Requirement-Topology table $(C)RRTo_t$}
	\label{tab:crrtot}
\end{table}

\begin{table}[H]
	\centering
	\begin{tabular}{|p{4cm}|p{8cm}|}
			\hline
			\textbf{Requirement} & \textbf{Dependency} \\
			\hline
			Survive & HoverDep. \\
			\hline
			Be available & HoverDep. \\
			\hline
			Maintain accessibility & HoverDep. \\
			\hline
		\end{tabular}
		\caption{Reference Requirement-Dependency table $(C)RReqD_t$}
	\label{tab:crreqdt}
\end{table}

\begin{table}[H]
	\centering
	\begin{tabular}{|p{4cm}|p{8cm}|}
			\hline
			\textbf{Legacy System} & \textbf{Function} \\
			\hline
			Simulator UI & render. \\
			\hline
			Simulator Creator & accept\_input. \\
			\hline
		\end{tabular}
	\caption{Reference Legacy System-Function table $(C)RLSF_t$}
	\label{tab:crlsft}
\end{table}

\begin{table}[H]
	\centering
	\begin{tabular}{|p{4cm}|p{8cm}|}
			\hline
			\textbf{Legacy System} & \textbf{Topology} \\
			\hline
			& \\
			\hline
		\end{tabular}
	\caption{Reference Legacy System-Topology table $(C)RLST_t$}
	\label{tab:crlsTt}
\end{table}

\begin{table}[H]
	\centering
	\begin{tabular}{|p{4cm}|p{8cm}|}
			\hline
			\textbf{Relation} & \textbf{Dependency} \\
			\hline
			Simulator Data & SimDataDep \\
			\hline
			Define Before Simulate & DefBefSimDep. \\
			\hline
			Create Before Exist & CreateInfDep. \\
			\hline
			Parameters Input & ParInputDep. \\
			\hline
		\end{tabular}
	\caption{Reference Relation-Dependency table $(C)RRD_t$}
	\label{tab:crrdt}
\end{table}

\subsubsection*{Responsibilities Tables}

\begin{table}[H]
	\centering
	\begin{tabular}{|p{5cm}|p{7cm}|}
			\hline
			\textbf{Task} & \textbf{Description} \\
			\hline
			list\_information & List the information available from the current
			position.\\
			\hline
			access\_information & Access the selected information available in the
			current position.\\
			\hline
			create\_information & Create a new hovering information. \\
			\hline
			obtain\_nodes\_information & Get information of each mobile node of the
			system.\\
			\hline
			obtain\_hovering\_information & Get information of each hovering
			information of the system. \\
			\hline
			obtain\_communication\_links & Get information about current data exchange
			between mobile nodes. \\
			\hline
			start & Start the simulation. \\
			\hline
			stop & Stop the simulation. \\
			\hline
			pause & Pause the simulation. \\
			\hline
			walk & Move around, basing on behaviour and environment. \\
			\hline
			survive & Jumps or replicates to mobile nodes in order to keep an high
			survivability. \\
			\hline
			be\_available & Jumps or replicates to mobile nodes in order to keep an
			high availability. \\
			\hline
			maintain\_accessibility & Jumps or replicates to mobile nodes in order to
			keep an high accessibility. \\
			\hline
			create\_people & Create and collocate a new person instance. \\
			\hline
		\end{tabular}
	\caption{Task table $(C)T_t$}
	\label{tab:ctt}
\end{table}

\begin{table}[H]
	\centering
	\begin{tabular}{|p{5cm}|p{7cm}|}
			\hline
			\textbf{Function} & \textbf{Description} \\
			\hline
			communicate\_data & Send data to a mobile node in range.\\
			\hline
			show\_information & Output the requested hovering information data.\\
			\hline
			discover\_neighbor & Find reachable mobile nodes.\\
			\hline
			insert\_information & Input from user data needed for a new hovering
			information.\\
			\hline
			inquire\_node & Get all the information about a mobile node.\\
			\hline
			inquire\_hovering\_information & Get all the information about a piece of
			hovering information. \\
			\hline
			render & Show the simulation data. \\
			\hline
			replicate & Copy an information to another mobile node. \\
			\hline
			jump & Move an information to another mobile node. \\
			\hline
			accept\_input & Accept initial simulation parameters as user input. \\
			\hline
		\end{tabular}
	\caption{Function table $(C)F_t$}
	\label{tab:cft}
\end{table}

\subsubsection{Topologies Tables}

\begin{table}[H]
	\centering
	\begin{tabular}{|p{4cm}|p{8cm}|}
			\hline
			\textbf{Topology} & \textbf{Description} \\
			\hline
			Anchor Area & Area associated to each hovering information, defined as a
			circular area with center into the \emph{anchor location} and radius the
			\emph{anchor radius}.\\
			\hline
			Communication Range & The maximum effective distance of a \emph{p2p}
			mobile node communication. \\
			\hline
		\end{tabular}
	\caption{Topology table $(C)Top_t$}
	\label{tab:ctopt}
\end{table}

\begin{table}[H]
	\centering
	\begin{tabular}{|p{4cm}|p{8cm}|}
			\hline
			\textbf{Task} & \textbf{Topology} \\
			\hline
			list\_information & Anchor Area, Communication Range. \\
			\hline
			access\_information& Anchor Area, Communication Range.\\
			\hline
			create\_information & Anchor Area, Communication Range.\\
			\hline
			survive & Anchor Area, Communication Range. \\
			\hline
			be\_available & Anchor Area, Communication Range. \\
			\hline
			maintain\_accessibility & Anchor Area, Communication Range. \\
			\hline
		\end{tabular}
		\caption{Task-Topology table $(C)TTop_t$}
	\label{tab:cttopt}
\end{table}

\begin{table}[H]
	\centering
	\begin{tabular}{|p{4cm}|p{8cm}|}
			\hline
			\textbf{Function} & \textbf{Topology} \\
			\hline
			communicate\_data & Communication Range.\\
			\hline
			discover\_neighbor & Communication Range.\\
			\hline
			insert\_information & Communication Range. \\
			\hline
			replicate & Anchor Area, Communication Range. \\
			\hline
			jump & Anchor Area, Communication Range. \\
			\hline
		\end{tabular}
	\caption{Function-Topology table $(C)FTop_t$}
	\label{tab:cftopt}
\end{table}


\subsubsection{Dependency Tables}

\begin{table}[H]
	\centering
	\begin{tabular}{|p{4cm}|p{8cm}|}
			\hline
			\textbf{Dependency} & \textbf{Description} \\
			\hline
			SimDataDep & access to all information about hovering system
			components. \\
			\hline
			DefBefSimDep & define simulation parameters before starting the
			simulation. \\
			\hline
			CreateInfDep & information must be created before performing any
			operation. \\
			\hline
			ParInputDep & Simulation parameters have to be inserted into the
			simulation system. \\
			\hline
			SimStartedDep & Simulation should be started. \\
			\hline
		\end{tabular}
	\caption{Dependency table $(C)D_t$}
	\label{tab:cdt}
\end{table}

\begin{table}[H]
	\centering
	\begin{tabular}{|p{5cm}|p{7cm}|}
			\hline
			\textbf{Task} & \textbf{Dependency} \\
			\hline
			list\_information & SimDataDep, SimStartedDep.\\
			\hline
			access\_information & CreateInfDep, SimStartedDep.\\
			\hline
			create\_information & CreateInfDep, SimDataDep, ParInputDep. \\
			\hline
			obtain\_nodes\_information & SimStartedDep, SimDataDep.\\
			\hline
			obtain\_hovering\_information & SimStartedDep, SimDataDep. \\
			\hline
			obtain\_communication\_links & SimStartedDep, SimDataDep. \\
			\hline
			start & DefBefSimDep, SimStartedDep. \\
			\hline
			stop & SimStartedDep. \\
			\hline
			pause &SimStartedDep. \\
			\hline
			walk & SimStartedDep. \\
			\hline
			survive & SimStartedDep, CreateInfDep. \\
			\hline
			be\_available & SimStartedDep, CreateInfDep. \\
			\hline
			maintain\_accessibility & SimStartedDep, CreateInfDep. \\
			\hline
			create\_people & DefBefSimDep, ParInputDep. \\
			\hline
		\end{tabular}
	\caption{Task-Dependency table $(C)TD_t$}
	\label{tab:ctdt}
\end{table}

\begin{table}[H]
	\centering
	\begin{tabular}{|p{5cm}|p{7cm}|}
			\hline
			\textbf{Function} & \textbf{Dependency} \\
			\hline
			communicate\_data & SimStartedDep, SimDataDep.\\
			\hline
			discover\_neighbor & SimStartedDep, SimDataDep.\\
			\hline
			show\_information & SimStartedDep, CreateInfDep.\\
			\hline
			insert\_information & SimStartedDep, SimDataDep.\\
			\hline
			inquire\_node & SimStartedDep, SimDataDep.\\
			\hline
			inquire\_hovering\_information & SimStartedDep, SimDataDep, CreateInfDep.\\
			\hline
			render & SimStartedDep, SimDataDep. \\
			\hline
			replicate & SimStartedDep, CreateInfDep. \\
			\hline
			jump & SimStartedDep, CreateInfDep. \\
			\hline
			accept\_input & DefBefSimDep. \\
			\hline
		\end{tabular}
	\caption{Function-Dependency table $(C)FD_t$}
	\label{tab:cfdt}
\end{table}

\begin{table}[H]
	\centering
	\begin{tabular}{|p{4cm}|p{8cm}|}
			\hline
			\textbf{Topology} & \textbf{Dependency} \\
			\hline
			Anchor Area & SimDataDep, CreateInfDep.\\
			\hline
			Communication Range & SimDataDep. \\
			\hline
		\end{tabular}
	\caption{Topology-Dependency table $(C)TopD_t$}
	\label{tab:ctopdt}
\end{table}

\subsection{Architectural Design}

\subsection{Detailed Design}

