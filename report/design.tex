\section{Hovering Information and Social System analysis and design}
\label{sec:design}

The system should implement the hovering information system working inside
a social environment. Mobile nodes are owned by people, who move inside an
environment composed by anchors, that is locations where pieces of hovering
information are present. Anchors are usually binded to points of interest, but
generally hovering information can be dynamically created by people.

Mobile nodes lose power and may have not enough energy to supply the whole
function.  In that case some mobile node features may be limited such as
information storage, communication, etc..

The system should simulate an hovering information usage, inside an
environment.  People, carrying mobile nodes, have to walk with different
behaviour, emulating movements inside an area composed by points of interest.
The simulation should gather periodic data about hovering information status
and properties (i.e.\ availability, survivability and accessibility), nodes
position and communications link.

\subsection{Preliminary analysis}

Requirements implicate that the system is composed by three sub-systems:
\begin{enumerate}
	\item Simulator (graphic interface and data analyzer).
	\item Hovering information (composed by pieces of hovering information and
		mobile nodes).
	\item Social system (a group of hovering information users moving into
		the environment).
\end{enumerate}

Aside from required interfaces, these subsystems can be designed independently
from each other.

A simulator user may want to create people and assign a behaviour, nodes and
initial hovering information.

A person behaviour can be initialized defining social roles. Taking the cue from
\cite{human} some behaviour can be defined:

\begin{description}
	\item[Guard:] a person who walks following a predefined path.
	\item[Owner:] a person who resides, works near a certain point of interest.
	\item[Ant:] a person who performs a random walk but he's influenced by other
		people movement.
	\item[Group:] some people walking together sharing one of the previous
		behaviour.
\end{description}

\subsection{Requirements Analysis}

\subsubsection*{Requirements Tables}

The subsystem \emph{Simulator} doesn't require independent control or
intelligence, so the simulator interface it can be assumed as a \emph{Legacy
System}, already present in the environment.

\begin{table}[H]
	\centering
	\begin{tabular}{|p{4cm}|p{8cm}|}
			\hline
			\textbf{Actor} & \textbf{Description} \\
			\hline
			Simulation Analyst & The simulation data analyzer. \\
			\hline
			Person & User of a mobile node. \\
			\hline
		\end{tabular}
	\caption{Actor table $(C)Ac_t$}
	\label{tab:cact}
\end{table}

\begin{table}[H]
	\centering
	\begin{tabular}{|p{4cm}|p{8cm}|}
			\hline
			\textbf{Requirement} & \textbf{Description} \\
			\hline
			Access Information & Access all the information that reside inside an anchor area. \\
			\hline
			Create Information & Creates a new hovering information. \\
			\hline
			Obtain System Data & Known the current data (position, information access,
			etc.) of all the system component. \\
			\hline
			Manage Simulation & Manage the simulation. \\
			\hline
		\end{tabular}
	\caption{Requirement table $(C)Re_t$}
	\label{tab:cact}
\end{table}

\begin{table}[H]
	\centering
	\begin{tabular}{|p{4cm}|p{8cm}|}
			\hline
			\textbf{Actor} & \textbf{Requirement} \\
			\hline
			Person & Access Information, Create Information. \\
			\hline
			Simulation Analyst & Obtain System Data, Manage Simulation. \\
			\hline
		\end{tabular}
	\caption{Actor-Requirement table $(C)AR_t$}
	\label{tab:cart}
\end{table}

\subsubsection*{Domain Tables}

\begin{table}[H]
	\centering
	\begin{tabular}{|p{4cm}|p{8cm}|}
			\hline
			\textbf{External Environment} & \textbf{Legacy system} \\
			\hline
			External & Simulator UI. \\
			\hline
		\end{tabular}
	\caption{External Environment-Legacy System table $(C)EELS_t$}
	\label{tab:ceelst}
\end{table}

\begin{table}[H]
	\centering
	\begin{tabular}{|p{4cm}|p{8cm}|}
			\hline
			\textbf{Legacy System} & \textbf{Description} \\
			\hline
			Simulator UI & Simulation interface: show simulation data. \\
			\hline
		\end{tabular}
	\caption{External Environment-Legacy System table $(C)EELS_t$}
	\label{tab:ceelst}
\end{table}

\subsubsection*{Relations Tables}

\begin{table}[H]
	\centering
	\begin{tabular}{|p{4cm}|p{8cm}|}
			\hline
			\textbf{Relation} & \textbf{Description} \\
			\hline
			Simulator Data & make relevant information available to the Simulator UI. \\
			\hline
		\end{tabular}
	\caption{Relation table $(C)Rel_t$}
	\label{tab:crelt}
\end{table}

\begin{table}[H]
	\centering
	\begin{tabular}{|p{4cm}|p{8cm}|}
			\hline
			\textbf{Requirement} & \textbf{Relation} \\
			\hline
			Access Information & Simulator Data. \\
			\hline
			Create Information & Simulator Data. \\
			\hline
			Obtain System Data & Simulator Data. \\
			\hline
			Manage Simulation & Simulator Data. \\
			\hline
		\end{tabular}
	\caption{Requirement-Relation table $(C)RR_t$}
	\label{tab:crrt}
\end{table}

\begin{table}[H]
	\centering
	\begin{tabular}{|p{4cm}|p{8cm}|}
			\hline
			\textbf{Legacy-System} & \textbf{Relation} \\
			\hline
			Simulator UI & Simulator Data. \\
			\hline
		\end{tabular}
	\caption{Relation-LegacySystem table $(C)RLS_t$}
	\label{tab:crlst}
\end{table}

\subsection{Analysis}

%\subsubsection*{References Tables}
%
%\begin{table}[H]
%	\centering
%	\begin{tabular}{|p{4cm}|p{8cm}|}
%			\hline
%			\textbf{Requirement} & \textbf{Task} \\
%			\hline
%			Access Information & list_information, access_information. \\
%			\hline
%			Create Information & create_information. \\
%			\hline
%			Obtain System Data & obtain_node_position, obtain_hovering_information,
%			obtain_communication_link. \\
%			\hline
%			Manage Simulation & start, stop, pause. \\
%			\hline
%		\end{tabular}
%	\caption{Reference Requirement-Task table $(C)RRT_t$}
%	\label{tab:crrtt}
%\end{table}
%
%\begin{table}[H]
%	\centering
%	\begin{tabular}{|p{4cm}|p{8cm}|}
%			\hline
%			\textbf{Requirement} & \textbf{Function} \\
%			\hline
%			Access Information & communicate_data, show_information, discover_neighbor \\
%			\hline
%			Create Information & communicate_data, insert_information, discover_neighbor \\
%			\hline
%			Obtain System Data & inquire_node, inquire_hovering_information. \\
%			\hline
%			Manage Simulation & render. \\
%			\hline
%		\end{tabular}
%	\caption{Reference Requirement-Function table $(C)RRF_t$}
%	\label{tab:crrft}
%\end{table}
%
%\begin{table}[H]
%	\centering
%	\begin{tabular}{|p{4cm}|p{8cm}|}
%			\hline
%			\textbf{Requirement} & \textbf{Topology} \\
%			\hline
%			& \\
%			\hline
%		\end{tabular}
%	\caption{Reference Requirement-Topology table $(C)RRTo_t$}
%	\label{tab:crrtot}
%\end{table}
%
%% TODO: mancano un po' di tabelle che non ho idea di cosa siano :/
%
%\subsubsection*{Responsibilities Tables}
%
%\begin{table}[H]
%	\centering
%	\begin{tabular}{|p{4cm}|p{8cm}|}
%			\hline
%			\textbf{Task} & \textbf{Description} \\
%			\hline
%			Manage device & Managing of device resources and features: communication,
%			neighbor discovery, geo-location, power management and storing
%			management. \\
%			\hline
%			Obtain information & An existing information has to be accessed. \\
%			\hline
%			Create Information & A new information, with all the required data has to
%			be inserted int the system. \\
%			\hline
%			Obtain system data & . \\
%			\hline
%			Be avaiable & Replicate and move information inside the information anchor area. \\
%			\hline
%			Survive & Replicate and move information keeping it alive. \\
%			\hline
%			Maintain accessibility & Replicate and move information in order to be
%			available in the whole anchor area. \\
%			\hline
%		\end{tabular}
%	\caption{Reference Requirement-Task table $(C)RRT_t$}
%	\label{tab:crrtt}
%\end{table}
%
%\begin{table}[H]
%	\centering
%	\begin{tabular}{|p{4cm}|p{8cm}|}
%			\hline
%			\textbf{Function} & \textbf{Description} \\
%			\hline
%			Communicate & Manage data transmission and receiving.  \\
%			\hline
%			Discover & Find other devices inside the communication range. \\
%			\hline
%			Management power & Check power usage and disable some functionality. \\
%			\hline
%			Management storage & Manage the memory, enabling or disabling information storing. \\
%			\hline
%		\end{tabular}
%	\caption{Reference Requirement-Task table $(C)RRT_t$}
%	\label{tab:crrtt}
%\end{table}
%
\subsection{Architectural Design}

\subsection{Detailed Design}

